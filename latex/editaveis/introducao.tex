\chapter[Introdução]{Introdução}

Neste capítulo, serão descritos a contextualização, apresentando brevemente o
tema; a justificativa, apresentando os porquês da elaboração do trabalho; a questão de
pesquisa; os objetivos geral e específicos, e a organização dos capítulos dessa monografia.

\section{CONTEXTUALIZAÇÃO}

O Brasil é amplamente reconhecido como o “país do futebol”, dada a enorme popularidade deste esporte em todo o território nacional. Estima-se que aproximadamente 70\% dos brasileiros o considerem seu esporte favorito \cite{gente2023}, e cerca de 30 milhões o pratiquem regularmente ou de forma ocasional \cite{sportbay2022}. Essa popularidade transcende as competições profissionais e se manifesta no cotidiano da população, por meio das tradicionais “peladas” entre amigos, jogos em quadras públicas e torneios amadores. Assim, o futebol não é apenas um passatempo, mas um elemento central da identidade cultural brasileira, promovendo socialização, lazer e senso de pertencimento.

No contexto do Distrito Federal, essa realidade não é diferente. O futebol de rua e o futsal se consolidaram como práticas comuns entre crianças, jovens e adultos. O futsal, em especial, está entre as quatro modalidades mais praticadas do país há décadas, com registros de mais de 10 milhões de praticantes ocasionais já no início dos anos 2000 \cite{foothub2022}. Em Brasília, a tradição das “peladas” se mantém viva por meio da ocupação espontânea das diversas quadras poliesportivas espalhadas pela cidade – muitas delas integradas ao projeto urbanístico das superquadras ou situadas em parques e centros comunitários. O governo local tem investido constantemente na manutenção e melhoria dessas estruturas \cite{segov2023}, ampliando o acesso à prática esportiva e incentivando hábitos saudáveis.

Essas quadras públicas representam espaços democráticos, acessíveis a todas as classes sociais, nos quais acontecem desde jogos informais até atividades de escolinhas e projetos sociais. Em horários de maior demanda, como fins de tarde ou finais de semana, os jogos são organizados de maneira colaborativa entre os participantes, muitas vezes utilizando grupos de WhatsApp ou redes sociais para combinar os encontros. Em alguns casos, especialmente em regiões como o Plano Piloto, foram implementadas regras formais de uso, com faixas de horário reservadas a escolas ou projetos sociais, exigindo agendamento junto à administração pública para outros horários \cite{correio2023}. Mesmo com essas medidas, a organização dos jogos segue, em sua maioria, informal e baseada no diálogo direto entre os jogadores.

Nesse cenário, surge a oportunidade de aliar tecnologia e tradição. A crescente presença de smartphones e o uso cotidiano de ferramentas digitais abrem espaço para soluções que otimizem a vivência do futebol amador. A proposta deste trabalho é justamente apresentar o desenvolvimento de um aplicativo móvel que facilite a organização de partidas de futebol entre amigos, utilizando as quadras públicas disponíveis em Brasília.

\section{OBJETIVOS}

Seguem os objetivos geral e específicos atrelados a esse trabalho.

\subsection{Objetivo geral}

Desenvolver um aplicativo multiplataforma para apoiar a organização de partidas de futebol amador em Brasília, facilitando o encontro entre jogadores, a gestão de quadras e a comunicação comunitária, por meio de funcionalidades intuitivas e acessíveis. O projeto busca, adicionalmente, ampliar a realização de partidas, fomentar o engajamento social, contribuir para a revitalização dos espaços públicos esportivos e promover a qualidade de vida da população.

\subsection{Objetivos específicos}

Os objetivos específicos são:

\begin{itemize}
    \item Objetivo específico 1
    \item Objetivo específico 2
    \item Objetivo específico 3
\end{itemize}