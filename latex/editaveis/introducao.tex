\chapter[Introdução]{Introdução}

Neste capítulo, serão descritos a contextualização, apresentando brevemente o
tema; a justificativa, apresentando os porquês da elaboração do trabalho; a questão de
pesquisa; os objetivos geral e específicos, e a organização dos capítulos dessa monografia.

\section{CONTEXTUALIZAÇÃO}

O Brasil é amplamente reconhecido como o “país do futebol”, dada a enorme popularidade deste esporte em todo o território nacional. Estima-se que aproximadamente 70\% dos brasileiros o considerem seu esporte favorito \cite{gente2023}, e cerca de 30 milhões o pratiquem regularmente ou de forma ocasional \cite{sportbay2022}. Essa popularidade transcende as competições profissionais e se manifesta no cotidiano da população, por meio das tradicionais “peladas” entre amigos, jogos em quadras públicas e torneios amadores. Assim, o futebol não é apenas um passatempo, mas um elemento central da identidade cultural brasileira, promovendo socialização, lazer e senso de pertencimento.

No contexto do Distrito Federal, essa realidade não é diferente. O futebol de rua e o futsal se consolidaram como práticas comuns entre crianças, jovens e adultos. O futsal, em especial, está entre as quatro modalidades mais praticadas do país há décadas, com registros de mais de 10 milhões de praticantes ocasionais já no início dos anos 2000 \cite{foothub2022}. Em Brasília, a tradição das “peladas” se mantém viva por meio da ocupação espontânea das diversas quadras poliesportivas espalhadas pela cidade – muitas delas integradas ao projeto urbanístico das superquadras ou situadas em parques e centros comunitários. O governo local tem investido constantemente na manutenção e melhoria dessas estruturas \cite{segov2023}, ampliando o acesso à prática esportiva e incentivando hábitos saudáveis.

Essas quadras públicas representam espaços democráticos, acessíveis a todas as classes sociais, nos quais acontecem desde jogos informais até atividades de escolinhas e projetos sociais. Em horários de maior demanda, como fins de tarde ou finais de semana, os jogos são organizados de maneira colaborativa entre os participantes, muitas vezes utilizando grupos de WhatsApp ou redes sociais para combinar os encontros. Em alguns casos, especialmente em regiões como o Plano Piloto, foram implementadas regras formais de uso, com faixas de horário reservadas a escolas ou projetos sociais, exigindo agendamento junto à administração pública para outros horários \cite{peixoto2025}. Mesmo com essas medidas, a organização dos jogos segue, em sua maioria, informal e baseada no diálogo direto entre os jogadores.

Nesse cenário, surge a oportunidade de aliar tecnologia e tradição. A crescente presença de smartphones e o uso cotidiano de ferramentas digitais abrem espaço para soluções que otimizem a vivência do futebol amador. A proposta deste trabalho é justamente apresentar o desenvolvimento de um aplicativo móvel que facilite a organização de partidas de futebol entre amigos, utilizando as quadras públicas disponíveis em Brasília.

\section{JUSTIFICATIVA}

A organização de partidas de futebol amador em Brasília enfrenta desafios recorrentes, especialmente nas quadras públicas distribuídas pela cidade. A marcação dos jogos costuma depender de grupos dispersos em aplicativos de mensagens, com confirmações inconsistentes e pouca clareza sobre a disponibilidade dos espaços. Essa dificuldade, relatada por jogadores de diferentes regiões do Distrito Federal, revela uma fragilidade na forma como atividades esportivas comunitárias são articuladas, mesmo diante de uma infraestrutura urbana que favorece esse tipo de prática. Ao ampliar essa análise, percebe-se que o problema ultrapassa a dimensão organizacional e se conecta a questões estruturais de saúde pública e urbanismo.

O Brasil apresenta índices alarmantes de sedentarismo. De acordo com o Ministério da Saúde, cerca de 47\% dos adultos brasileiros não praticam atividade física suficiente, e entre os jovens esse índice chega a 84\% \cite{ministerio2023,saraiva2024}. Esses números colocam o país como o mais sedentário da América Latina e o quinto no mundo em inatividade física \cite{oms2023,bvsms2023}. O sedentarismo é considerado atualmente um dos maiores desafios de saúde pública, estando associado ao aumento de doenças crônicas como obesidade, diabetes tipo 2, hipertensão, além de transtornos mentais \cite{bvsms2023,oms2023}. Estima-se que a inatividade física seja responsável por aproximadamente 5 milhões de mortes anuais em todo o mundo \cite{oms2023}.

O estilo de vida contemporâneo, marcado pelo uso excessivo de tecnologias, transporte passivo e longos períodos em frente a telas, contribui significativamente para essa inatividade \cite{saraiva2024}. Comparando dados históricos, um adulto jovem caminhava cerca de 10 mil passos por dia na década de 1970, enquanto hoje essa média caiu para aproximadamente 2 mil \cite{saraiva2024}. Entre adolescentes, o tempo de tela elevado também tem sido correlacionado ao aumento da obesidade e problemas posturais \cite{saraiva2024}.

Nesse contexto, práticas esportivas comunitárias ganham importância por sua relevância. Além de promoverem gasto calórico, melhoras fisiológicas e bem-estar mental, elas oferecem um ambiente de apoio social mútuo que favorece a adesão à atividade física \cite{ministerio2023}. Esportes coletivos como o futebol, ao criarem vínculos entre os praticantes, funcionam como ferramentas para vencer o sedentarismo e incentivar rotinas ativas \cite{ministerio2023,carbinatto2022}.

Brasília conta com uma vasta rede de quadras públicas integradas ao seu projeto urbanístico, muitas delas distribuídas pelas superquadras residenciais \cite{segov2023}. No entanto, esses espaços nem sempre são plenamente utilizados. A ausência de mecanismos de organização faz com que partidas deixem de acontecer, mesmo quando há estrutura e demanda local \cite{ipea2021}.

No caso do Plano Piloto, por exemplo, a regulamentação vigente determina que, de segunda a sexta-feira, as quadras sejam priorizadas para escolas e projetos sociais durante o dia, ficando abertas à comunidade apenas entre 20h e 22h \cite{peixoto2025}\footnote{Essa regulamentação estabelece que, caso um projeto social com autorização formal esteja em andamento, ele terá prioridade no uso da quadra, podendo inclusive solicitar a saída de usuários presentes. Contudo, tais situações são pouco frequentes, uma vez que até mesmo projetos sociais têm presença escassa nas quadras públicas.}. Essa informação, porém, nem sempre é divulgada de forma clara, dificultando o planejamento da comunidade. Além disso, sem um sistema integrado de agendamento e visualização, muitas quadras permanecem vazias, enquanto outras se tornam objeto de conflito devido à alta demanda \cite{agenciabrasilia2025}.

Outro desafio recorrente é a dificuldade de reunir o número ideal de jogadores. A comunicação informal, baseada em mensagens avulsas em grupos de WhatsApp, muitas vezes resulta em partidas desmarcadas, desistências de última hora e exclusão de novos participantes. Pessoas recém-chegadas à cidade ou fora dos círculos de amizade locais enfrentam barreiras para encontrar partidas, o que desestimula a prática esportiva comunitária.

Diante dessa problemática, a proposta deste trabalho é o desenvolvimento de uma solução mobile voltada à organização do futebol amador em Brasília. A aplicação centraliza informações sobre partidas, jogadores e disponibilidade de quadras, permitindo agendamento, confirmação de presença e gerenciamento de listas, de modo a otimizar o uso dos espaços públicos e incentivar a prática esportiva local.

Sob o ponto de vista acadêmico, o projeto também se justifica por proporcionar uma aplicação prática dos fundamentos da Engenharia de Software a um problema social real. Seu desenvolvimento envolve levantamento e análise de requisitos, escolha de tecnologias multiplataforma, definição de arquitetura, implementação iterativa e validação com usuários reais. O produto final não apenas oferece impacto comunitário mensurável, como também contribui para a formação técnica e científica de seus autores, documentando métodos e resultados com potencial de replicação em outras regiões e contextos.

\section{OBJETIVOS}

Seguem os objetivos geral e específicos atrelados a esse trabalho.

\subsection{Objetivo geral}

Desenvolver um aplicativo multiplataforma para apoiar a organização de partidas de futebol amador em Brasília, facilitando o encontro entre jogadores, a gestão de quadras e a comunicação comunitária, por meio de funcionalidades intuitivas e acessíveis. O projeto busca, adicionalmente, ampliar a realização de partidas, fomentar o engajamento social, contribuir para a revitalização dos espaços públicos esportivos e promover a qualidade de vida da população.

\subsection{Objetivos específicos}

Para alcançar o objetivo geral proposto, o desenvolvimento do aplicativo será conduzido por meio dos seguintes objetivos específicos, que representam as principais etapas e práticas adotadas ao longo do processo:

\begin{itemize}
    \item Realizar o levantamento e análise de requisitos por meio de entrevistas, brainstorms, observação de campo e histórias de usuário.
    \item Elaborar e priorizar o backlog do produto, definindo a ordem de implementação com base em valor para o usuário e viabilidade técnica.
    \item Produzir protótipos de baixa e alta fidelidade e conduzir testes de usabilidade para validar fluxos, arquitetura de informação e experiência de uso antes da implementação.
    \item Definir a arquitetura da aplicação, incluindo camadas de frontend, backend e banco de dados, bem como tecnologias e padrões de projeto.
    \item Modelar e implementar o banco de dados, garantindo estrutura lógica eficiente, integridade e suporte às funcionalidades planejadas.
    \item Implementar o sistema de forma iterativa, seguindo boas práticas de desenvolvimento.
    \item Conduzir testes funcionais, de integração e de qualidade no produto final, validando desempenho e robustez.
    \item Documentar todo o processo de desenvolvimento, registrando decisões técnicas, desafios enfrentados e soluções aplicadas.
    \item Avaliar os resultados obtidos em relação aos objetivos iniciais, identificando oportunidades de melhoria e perspectivas futuras.
\end{itemize}